% Ensure included PDFs (banners) with newer versions embed without warnings
\pdfminorversion=7
\documentclass[aspectratio=169]{beamer}

\makeatletter
% Make LaTeX find theme .sty files in Template
\def\input@path{{../Template/}}
\makeatother
\usepackage{graphicx}
% Make graphics (including banner PDFs) resolvable without TEXINPUTS
\graphicspath{{../Template/}{../Template/banners/}{../../Common_Images/}}

%================================================================%
% Theme and Package Setup
%================================================================%
\usetheme{ucl}
\setbeamercolor{banner}{bg=darkpurple}

% Navigation and Footer
\setbeamertemplate{navigation symbols}{\vspace{-2ex}}
\setbeamertemplate{footline}[author title date]
\setbeamertemplate{slide counter}[framenumber/totalframenumber]

\usepackage[utf8]{inputenc}
\usepackage[british]{babel} % British spelling
\usepackage{amsmath, amssymb, amsthm}
\usepackage{graphicx}
\usepackage{tikz}
\usetikzlibrary{calc, positioning, arrows.meta, shapes.geometric, trees, backgrounds, shapes.misc, graphs, quotes, shadows} 
\usefonttheme{professionalfonts}
\usepackage{eulervm}
\usepackage{booktabs}
\usepackage{listings}
\usepackage{xcolor}

% Define custom colours
\makeatletter
\@ifundefined{color@stone}{%
    \definecolor{stone}{gray}{0.95}%
}{}
\makeatother
\definecolor{darkgreen}{rgb}{0.0, 0.5, 0.0}

\setbeamercovered{transparent}

% Configure listings for Python and C
\lstset{
  basicstyle=\ttfamily\scriptsize, % Smaller font for code to fit
  keywordstyle=\color{blue},
  commentstyle=\color{gray!80!black},
  stringstyle=\color{darkgreen},
  showstringspaces=false,
  breaklines=true,
  frame=single,
  backgroundcolor=\color{stone},
  numbers=left,
  numberstyle=\tiny\color{gray},
  escapeinside={(*@}{@*)},
  tabsize=4
}

% Define theoremblock
\makeatletter
\newenvironment<>{theoremblock}[1]{%
    \begin{block}#2{#1}%
}{\end{block}}
\makeatother

% Section divider slides
\AtBeginSection[]{
  \begin{frame}
    \frametitle{\textbf{\Large\insertsectionhead}}
    \begin{center}
        \huge\textbf{\insertsectionhead}
    \end{center}
  \end{frame}
}

\title{Introduction to COMP0263}
\subtitle{What to expect from this module}
\author[Martin Benning (University College London)]{Martin Benning}
\date[COMP0263]{COMP0263 -- Solving Inverse Problems with Data-Driven Models \\[1cm] 13 January 2026}
\institute[]{University College London}

\begin{document}

% --- Title Frame ---
\begin{frame}
  \titlepage
\end{frame}

% --- Overview ---
\begin{frame}
\frametitle{Lecture Overview}
\begin{block}{Objectives}
    \begin{itemize}
        \item Understand the logistics: Moodle, GitHub, and Scheduling.
        \item Define the scope of COMP0263 vs. COMP0114.
        \item Outline the assessment structure (Formative assessment, Courseworks).
        \item Introduce the \textit{data-driven} philosophy.
    \end{itemize}
\end{block}
\tableofcontents
\end{frame}

% --- Teaching Team ---
\begin{frame}
\frametitle{Your Teaching Team}
\begin{columns}[T]
    \begin{column}{0.5\textwidth}
        \centering
        \includegraphics[width=0.5\textwidth]{martin_benning.jpg}
        
        \vspace{0.5cm}
        \textbf{Prof. Martin Benning}
        
        \textit{Module Instructor}
        
        \vspace{0.1cm}
        {\tiny Office Hours: Fridays, 12:30 - 13:30 (from next week)}
        
        \vspace{0.1cm}
        \small \href{https://profiles.ucl.ac.uk/95169-martin-benning}{UCL Profile}
    \end{column}
    
    \begin{column}{0.5\textwidth}
        \centering
        \includegraphics[width=0.5\textwidth]{serban_tudosie.jpg}
        
        \vspace{0.5cm}
        \textbf{Serban Cristian Tudosie}
        
        \textit{Teaching Assistant}
        
        \vspace{0.1cm}
        {\tiny \phantom{Office Hours: Fridays, 12:30 - 13:30 (from next week)}}
        
        \vspace{0.1cm}
        \small \href{https://profiles.ucl.ac.uk/96775-serban-cristian-tudosie}{UCL Profile}
    \end{column}
\end{columns}
\end{frame}

\section{Logistics and Communication}

% \begin{frame}
\begin{frame}[fragile]
\frametitle{Communication Hubs}
\begin{itemize}
    \item \textbf{Moodle}: The primary point of contact.
    \item \textbf{GitHub}: The technical hub for content.
    \begin{itemize}
        \item \textbf{Lecture Notes}: An evolving, book-style document.
        \item \textbf{Slides}: Released weekly via GitHub Releases.
        \item \textbf{Discussions}: Use the Q\&A feature to upvote/downvote questions.
        \item \textbf{Pull Requests}: Found a typo? Fork the repo, fix it, and submit a PR.
    \end{itemize}
\end{itemize}

\begin{block}{Getting the Materials}
\begin{lstlisting}[language=bash]
# Clone the repository
git clone https://github.com/UCL-COMP0263-2026/Course-Materials.git
# Check Releases for pre-compiled PDFs!
\end{lstlisting}
\end{block}
\end{frame}

\begin{frame}
\frametitle{Weekly Schedule}
Each week consists of two lecture blocks and one tutorial block.

\begin{table}[]
\centering
\begin{tabular}{l l l l}
\toprule
\textbf{Day} & \textbf{Time} & \textbf{Activity} & \textbf{Location} \\ 
\midrule
\textbf{Tuesday} & 09:00 - 11:00 & Lecture Block 1 & IOE - Bedford Way (20) W3.04 \\
\textbf{Thursday} & 13:00 - 15:00 & Lecture Block 2 & Gower Street (66-72) G01 \\
\textbf{Friday} & 14:00 - 16:00 & Computer Practical & Birkbeck Malet St 422/423 \\ 
\bottomrule
\end{tabular}
\end{table}

\begin{itemize}
    \item \textbf{Lectures}: Split into $4 \times 50$ minute slots total.
    \item \textbf{Structure}: First 3 slots for slides/theory. Final slot is \textbf{Flipped Classroom} (Q\&A, Deep Dives).
    \item \textbf{Practicals}: TA-led. Focus on Coursework support and coding.
\end{itemize}
\end{frame}

\section{Module Content}

\begin{frame}
\frametitle{What is this module?}

\textbf{COMP0263} bridges the gap between mathematical inverse problem theory and modern deep learning.

\begin{alertblock}{The "Guinea Pig" Disclaimer}
This module is in its first year. You are the pioneer cohort.
\begin{itemize}
    \item There may be teething issues.
    \item Feedback is vital—tell us what works and what doesn't.
    \item We (Instructors \& TAs) will exercise leniency where necessary.
\end{itemize}
\end{alertblock}
\end{frame}

\begin{frame}
\frametitle{Relationship to COMP0114}
\small
\textbf{COMP0114 (Inverse Problems in Imaging)} focuses on "hand-crafted" model-based methods.
\begin{itemize}
    \item \textbf{Analytic Solutions}: Moore-Penrose pseudo-inverse and SVD analysis.
    \item \textbf{Variational Regularisation}: Explicit functionals like Tikhonov ($\Psi(f) = ||f||^2$) or Total Variation ($\Psi(f) = \int |\nabla f|$).
    \item \textbf{Iterative Solvers}: Gradient Descent, Landweber, and Conjugate Gradients.
    \item \textbf{Bayesian approach}: MAP estimates with fixed priors.
\end{itemize}
\vspace{1em}
\textbf{COMP0263} asks: \textit{What if we learn the regulariser, the parameters, or the entire solver from data?}
\end{frame}

\begin{frame}
\frametitle{Weekly Syllabus: Weeks 1-4}
\begin{itemize}
    \item \textbf{Week 1: Introduction} \\
    Defining ill-posedness, stability, and why standard inversions fail on real data.
    
    \item \textbf{Week 2: Model-based Regularisation} \\
    Recap of classical "hand-crafted" techniques (Spectral filtering, Variational optimisation) as a baseline.
    
    \item \textbf{Week 3: Data-driven Basics} \\
    Learning parameters from data. Bilevel optimisation and optimal filter derivation.
    
    \item \textbf{Week 4: Neural Regularisers} \\
    Replacing hand-crafted penalties with Deep Neural Networks (e.g., NETT framework).
    \item \textbf{Week 5: Unrolling Algorithms} \\
    Interpreting iterative algorithms as neural networks. Sparse recovery and denoising.
\end{itemize}
\end{frame}

\begin{frame}
\frametitle{Weekly Syllabus: Weeks 5-10}
\begin{itemize}
    \item \textbf{Week 6: Variational Networks \& Implicit Layers} \\
    Deep Equilibrium Models (DEQs) and infinite depth networks.
    \item \textbf{Week 7: Deep Learning as Optimal Control} \\
    Viewing ResNets as discretised ODEs. Pontryagin's principle and continuous-depth architectures.
    \item \textbf{Week 8: Lifted frameworks for training/inverting networks} \\
    Lifted Bregman training, avoiding backpropagation, and convex auxiliary problems.
    \item \textbf{Week 9: Frontiers of Data-Driven Regularisation} \\
    Generative priors, diffusion models, and emerging research directions.
    \item \textbf{Week 10: Guest Lectures.}
\end{itemize}
\end{frame}

\section{Assessment \& Policies}

\begin{frame}
\frametitle{AI Assistance Policy}
\begin{block}{UCL Category 2: AI tools can be used in an assistive role}
    For module content and courseworks (except CW3), AI assistance is \textbf{permitted} as a support tool.
\end{block}

\begin{itemize}
    \item \textbf{Assistive Role}: Use AI for coding help, debugging, or explaining concepts. You must still be the author of your own work.
    \item \textbf{Responsibility}: You are responsible for all outputs. Verify AI suggestions carefully as hallucinations are common.
    \item \textbf{Transparency}: Acknowledge AI usage in your submissions where appropriate.
    \item \textbf{Exception}: \textbf{Coursework 3} is a Category 1 assessment (No AI allowed).
\end{itemize}
\end{frame}

\begin{frame}
\frametitle{Assessment Structure}
\begin{enumerate}
    \item \textbf{Formative Assessments (0\%)}
    \begin{itemize}
        \item Released Week 1 \& 2, Due Week 2 \& 3.
        \item Designed to "get you going".
        \item AI allowed, but try to solve manually first to understand the trade.
    \end{itemize}
    
    \item \textbf{Coursework 1 (25\%)}
    \begin{itemize}
        \item Traditional computing tasks + Open-ended training task.
        \item Goal: Achieve a quality metric that untrained methods cannot.
    \end{itemize}
    
    \item \textbf{Coursework 2 (25\%)}
    \begin{itemize}
        \item \textbf{Deblurring Challenge}.
        \item Competitive ranking based on quality metrics on unseen data.
    \end{itemize}
\end{enumerate}
\end{frame}

\begin{frame}
\frametitle{Coursework 3 (Final Assessment)}
\begin{alertblock}{Coursework 3 (50\%) -- Category 1 Assessment}
    \begin{itemize}
        \item \textbf{Format}: In-person supervised session (Likely Tue, April 21).
        \item \textbf{Duration}: 3 hours.
        \item \textbf{Content}: Mixture of Multiple Choice Questions and Coding Exercises.
        \item \textbf{Restrictions}: \textbf{NO} Internet access. \textbf{NO} AI assistance (Category 1: GenAI tools cannot be used).
        \item \textit{Note: This ensures you understand the fundamental concepts without reliance on external tools.}
    \end{itemize}
\end{alertblock}
\end{frame}

\section{Next Steps}

\begin{frame}
\frametitle{Coming up this week...}
\begin{itemize}
    \item \textbf{Next}: Two more 50-minute blocks on \textit{Introduction to Inverse Problems}.
    \begin{itemize}
        \item What makes inverse problems ill-posed?
        \item Stability and generalised inverse fundamentals.
    \end{itemize}
    \item \textbf{Friday Practical}: First Computer Practical (14:00--16:00).
    \begin{itemize}
        \item Introduction to the Formative Assessment.
        \item Set up your coding environment.
    \end{itemize}
\end{itemize}
\end{frame}

\begin{frame}
\frametitle{Before next week, please...}
\begin{itemize}
    \item \textbf{Log in to Moodle} and check your access.
    \item \textbf{Star/Fork the GitHub Repository} to track updates.
    \item \textbf{Download the Lecture Notes} (Release v1.0).
    \item Familiarise yourself with the assessment structure.
\end{itemize}

\vspace{1cm}
\centering
\Large \textbf{Questions?}
\end{frame}

\end{document}